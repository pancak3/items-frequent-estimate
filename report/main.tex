\documentclass[10pt]{article}
\usepackage[utf8]{inputenc}
\usepackage[a4paper, left=25mm, top=25mm, right=25mm, bottom=25mm]{geometry}
\usepackage[utf8]{inputenc}
\usepackage{listings}
\usepackage{minted}
\usepackage[english]{babel,isodate}
\usepackage[pdftex, pdfauthor={Qifan Deng},
 pdftitle={COMP90056 - Stream Computing and Applications 2020, Assignment 2}, 
 pdfsubject={COMP90056 Assignment}]{hyperref}
\usepackage{pgfplots} 
\usepackage{SIunits}        % <-- required in preamble
\pgfplotsset{compat=newest} % 
\usepackage[justification=centering]{caption}
\usepackage{minted}
\usepackage{enumitem}
\usepackage{subcaption}
\usepackage{amsmath}
\usepackage[ruled,linesnumbered]{algorithm2e}
\usepackage{epstopdf}

\newcommand\gauss[2]{1/(#2*sqrt(2*pi))*exp(-((x-#1)^2)/(2*#2^2))}
\newcommand{\Mod}[1]{\ (\mathrm{mod}\ #1)}
\newcommand\numberthis{\addtocounter{equation}{1}\tag{\theequation}}

\setlength{\columnsep}{20pt}
\setlength{\parindent}{8pt}
\setlength{\parskip}{3pt}

\pgfplotsset{compat=1.16}

\title{COMP90056 - Stream Computing and Applications 2020, Assignment 2 
\\Frequent Items in a Data Stream}
\author{
  Qifan Deng (1077479)\\
  \texttt{qifand@student.unimelb.edu.au} }
\date{\printdayoff\normalsize\today}

\begin{document}
\sloppy
% \twocolumn
\maketitle

\section{Introduction}
Three algorithms are implemented in this report to estimate the most frequent items of a data stream.
They are StickySampling, LossyCounting and SpaceSaving. 
There is also a Baseline algorithm implemented to investigate the performance of the three algorithms.
Results of experiments are provided in Section? which show ? 

\section{Hardware \& Environments \& Data Stream}

\paragraph{Hardware}
The experiments in section are conducted on a machin with the following specs
\begin{itemize}
     \setlength\itemsep{1pt}
       \item CPU: 1.8 GHz Quad-Core Intel Core i5
       \item Memory: 8 GB 2133 MHz LPDDR3
       \item Disk: WDC PC SN720 SDAPNTW-512G-1127 SSD
\end{itemize}
\paragraph{Environment}
The testing system is macOS Catalina version 10.15.7 (19H2).
All the algorithms are implemented in Python 3.8.5.
The requirements of the algorithms are stored in requirements.txt which can be installed with command pip install -r requirements.txt.

\paragraph{Data Stream}
The data stream obeys power-law distribution where 
the $i^{th}$ most frequent item has probability $\frac{1}{i^{z} \cdot{} Zeta(z)}$
where $z$ is a positive real-value parameter and $Zeta$ is Riemann or Hurwitz zeta function ?. 
Figure~\ref{power-law} shows the distribution when $z = \{1.1, 1.4, 1.7, 2.0\}$ and the data stream size is $10^6$.


\begin{figure}
     \begin{subfigure}[b]{0.5\textwidth}
          \centering
          \resizebox{\linewidth}{!}{\includegraphics{eps/zipf-1.1-100-stream-1000000.eps}}
          \label{power-law-z-1.1-100-stream-1000000}
    \end{subfigure}
    \begin{subfigure}[b]{0.5\textwidth}
          \centering
          \resizebox{\linewidth}{!}{\includegraphics{eps/zipf-1.4-100-stream-1000000.eps}}
          \label{power-law-z-1.4-100-stream-1000000}
    \end{subfigure}
    \begin{subfigure}[b]{0.5\textwidth}
          \centering
          \resizebox{\linewidth}{!}{\includegraphics{eps/zipf-1.7-100-stream-1000000.eps}}
          \label{power-law-z-1.7-100-stream-1000000}
    \end{subfigure}
    \begin{subfigure}[b]{0.5\textwidth}
          \centering
          \resizebox{\linewidth}{!}{\includegraphics{eps/zipf-2.0-100-stream-1000000.eps}}
          \label{power-law-z-2.0-100-stream-1000000}
    \end{subfigure}
 
    \caption{Power-law distribution with $z = \{1.1, 1.4, 1.7, 2.0\}$ and data stream size $10^6$}
    \label{power-law}
\end{figure}

\section{Implementations}
\paragraph{Baseline}
Baseline algorithm is simple to implement. The algorithm holds counters $C$.
The $i^{th}$ item $x$ in data stream with $N$ items is called entry $<x, f>$. 
If $x$ is not in $C$ , set $C_n$ to $<x, 1>$. Otherwise, increment $C_x$ to $<x, f+1>$.
When user requests with support $s$, return the entries with $f$ greater than $f\cdot{}N$.
\paragraph{StickySampling \& LossyCounting \& SpaceSaving}
These three algorithms are implemented following the respective papers.
The theoretical performance of them should be that Table~\ref{theretical_performance} shows.
\paragraph{Theoretical Performance}
\begin{table}[h!]
     \centering
      \begin{tabular}{||c | c | c| c| c||} 
      \hline
      & Baseline & StickySampling & LossyCounting & SpaceSaving \\ [0.5ex] 
      \hline\hline
      Update Time & $O(1)$ & $O(n)$ & $O(n)$ & $O(log(n))$ \\
      \hline
      Memory & $O(n)$ & $O(n)$ & $O(n)$ & $O(1)$ \\ 
      \hline
      Accuracy & 100\% &545 & 778 & 7507 \\
      \hline
      \end{tabular}
     \caption{Theoretical Performance of Baseline, StickySampling, LossyCounting and SpaceSaving;
      $n$ is the number of distinct items }
      \label{theretical_performance}
\end{table}
     
\bibliographystyle{IEEEtran}
\bibliography{main}
\end{document}
